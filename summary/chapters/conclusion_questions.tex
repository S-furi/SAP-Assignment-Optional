%!TEX root = ../main.tex
\section{Insights}
In the following section some actionable insights that have been considered as
particularly relevant are proposed and described.

\paragraph{Modular agents: thinking beyond monoliths.} Agents should be
\emph{structured and modular}, meaning that a framework for language agents
would consolidate technical investment and improve compatibility, both in
academic research by means of standardised terms and open source implementation
facilitating modular plug-and-play and re-use, and in industry application
maintaining a single company-wide ''language agent library`` reducing technical
dept facilitating testing and re-use, enabling end users to experience a context-specific
instanciation of the same base agent.

\paragraph{Agent design: thinking beyond simple reasoning.} \ac{CoALA} defines
agents over three different concepts (internal memory, set of possible internal
and external actions and a decision making procedure), thus when developing an
application-specific agent it consists of specifying implementations for each
of these components. More in particular:
\begin{itemize}
    \item \textbf{What memory modules are necessary}
    \item \textbf{Define agent's internal action space}, primarily consisting
        of defining read and write access to each of the agent's memory
        modules.
    \item\textbf{Define decision-making procedure} specifying how reasoning
        and retrieval actions are taken in order to choose an external or learning
        actions. This step requires a tradeoff between performance and
        generalisation.
\end{itemize}

\paragraph{Structured reasoning: thinking beyond prompt engineering.} \ac{CoALA} suggests
a more structured reasoning procedure to update working memory variables:
\begin{itemize}
    \item \textbf{Prompting frameworks} like
        LangChain\footnote{https://www.langchain.com} can be used to define
        higher-level sequences of reasoning steps, reducing the burden of
        reasoning per \ac{LLM} call and the low-level prompt crafting efforts.
    \item \textbf{Reasoning usecases in agents can inform and reshape LLM
        training} in terms of the types and formats of training instances.
        \ac{LLM}s trained or finetuned beyond NLP tasks and adapted towards
        different capabilities will more likely be the backbone of future
        agents.
\end{itemize}

\paragraph{Long-term memory: thinking beyond retrieval augmentation.}
Memory-augmented agents can both read and write self-generated content
autonomously, opening the way for efficient lifelong learning mechanisms such
as combining existing human knowledge with new experience and skills in order
to help agents bootstrap to learn efficiently, with, for example, auto
generation of episodic memory gaining knowledge from experience and reason
about it. More in general, the possibility of adaptive mechanisms interleaving
memory search and \emph{forward simulation} will allow agents to make the most
of their knowledge.

\paragraph{Learning: thinking beyond in-context learning or finetuning} by means of:
\begin{itemize}
    \item \textbf{Meta-learning by modifying agent's code}, for example
        learning better retrieval procedures or learning to reason about when
        certain knowledge would be useful.
    \item \textbf{New forms of learning (and unlearning)} could include
        fine-tuning \emph{smaller models} for specific reasoning sub-tasks and
        studying the interaction effects between multiple forms of learning.
\end{itemize}

\paragraph{Action space: thinking beyond external tools or actions.} A good
portion of agent's design should face the decision of the size and complexity
of the action space: larger and complex action space are very well performing
in complex scenarios, but they must face a more complex and sophisticated
decision-making problem, most of the time relying on more customised or
hand-crafted decision proceudres. A tradeoff of the action space vs.
decision-making complexities is a basic problem to be considered before
development, and as a general rule, taking the minimal action space necessary
to solve a given task might be preferred.
A non-negligible problem to address is the \emph{safety} of the actions space
for some actions that are inherently riskier. Today safety measures are
typically task specific heuristics.

\paragraph{Decision Making: thinking beyond action generation.} Current agents
have just scratched the surface of a more deliberate, propose-evaluate-select
decision making procedures.

\begin{itemize}
    \item \textbf{Mixing language based reasoning and code-based planning} may
        offer the best of both worlds, where agents may write and execute
        simulation code on the fly to evaluate the consequences of plans and
        plan using \ac{LLM}s to translate from natural language to structured
        world models.

    \item \textbf{Metareasoning to improve efficiency} considering that
        \ac{LLM}s calls are both slow and computationally intensive, and that
        there must be considered a balance between cost (computational and not)
        and the utility of the resulting improved plan, future work should
        develop mechanism to estimate the utility of planning (as humans seems
        to do).
\end{itemize} 

\paragraph{Calibration and Alignment.} More complex decision making is
currently bottlenecked by issues such as \emph{over-confidence} and
miscalibration, misalignment with human values or bias, hallucinations in
self-evaluation and lack of \emph{human-in-the-loop} mechanisms to face
uncertainties.

